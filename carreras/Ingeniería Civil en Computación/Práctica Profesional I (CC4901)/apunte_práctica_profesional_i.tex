\documentclass[12pt]{article}

% --- Paquetes básicos ---
\usepackage[spanish]{babel}
\usepackage[utf8]{inputenc}
\usepackage[T1]{fontenc}
\usepackage{amsmath, amssymb, amsthm} % Matemáticas y teoremas
\usepackage{graphicx} % Insertar imágenes
\usepackage{hyperref} % Hipervínculos
\usepackage{geometry} % Márgenes
\geometry{margin=2.5cm}
\usepackage{xcolor} % Colores para resaltar

% --- Datos del documento ---
\title{Práctica Profesional I}
\author{FCFM Club, [Tu nombre aquí!]}
\date{\today}

\begin{document}

\maketitle
\tableofcontents
\newpage

% =============================
\section{Introducción}
Este documento contiene los apuntes del ramo Práctica Profesional I (CC4901).

Breve presentación del ramo, objetivos y contexto.

% =============================
\section{Tema 1: [Nombre del tema]}
Descripción y explicación del primer tema.

\subsection{Subtema 1.1: [Nombre del subtema]}
Detalle, definiciones, fórmulas, etc.

\subsubsection{Ejemplo}
Aquí se puede incluir un ejemplo resuelto.

\subsection{Subtema 1.2: [Nombre del subtema]}
Otro subtema relacionado.

% =============================
\section{Tema 2: [Nombre del tema]}
Segundo tema del ramo.

% Puedes seguir agregando temas y subtemas aquí.

% =============================
\section{Ejercicios propuestos}
\begin{enumerate}
    \item Ejercicio 1: Descripción del ejercicio.
    \item Ejercicio 2: Descripción del ejercicio.
          % Agrega más ejercicios según sea necesario.
\end{enumerate}

% =============================
\section{Referencias}
\begin{thebibliography}{9}
    \bibitem{ref1} Autor(es). \textit{Título}. Editorial, Año.
    % Agrega referencias según corresponda.
\end{thebibliography}

\end{document}
